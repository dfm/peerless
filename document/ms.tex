% This paper is part of the single transits project.
% Copyright 2015 Dan Foreman-Mackey (NYU) and the co-authors listed below.
%
%  RULES OF THE GAME
%
%  * 80 characters
%  * line breaks at the ends of sentences
%  * eqnarrys ONLY
%  * ``light curve'' not ``light-curve'' or ``lightcurve''
%  * Do not put in any comments that might get tweeted by @OverheardOnAph
%    (or maybe do put in a few....)
%  * ``percent'' (not \%) is a unit, as is ppm, so 5~percent.
%  * that is all.
%

\documentclass[12pt,preprint]{aastex}

\pdfoutput=1

\usepackage{color,hyperref}
\definecolor{linkcolor}{rgb}{0,0,0.5}
\hypersetup{colorlinks=true,linkcolor=linkcolor,citecolor=linkcolor,
            filecolor=linkcolor,urlcolor=linkcolor}
\usepackage{url}
\usepackage{amssymb,amsmath}
\usepackage{subfigure}
\usepackage{booktabs}

\usepackage{natbib}
\bibliographystyle{apj}

\newcommand{\project}[1]{\textsl{#1}}
\newcommand{\kepler}{\project{Kepler}}
\newcommand{\KT}{\project{K2}}
\newcommand{\tess}{\project{TESS}}
\newcommand{\jwst}{\project{JWST}}
\newcommand{\terra}{\project{TERRA}}
\newcommand{\pdc}{\project{PDC}}
\newcommand{\license}{MIT License}
\newcommand{\projectname}{\project{ketu}}

\newcommand{\paper}{\textsl{Article}}

\newcommand{\foreign}[1]{\emph{#1}}
\newcommand{\etal}{\foreign{et\,al.}}
\newcommand{\etc}{\foreign{etc.}}
\newcommand{\True}{\foreign{True}}
\newcommand{\Truth}{\foreign{Truth}}

\newcommand{\figref}[1]{\ref{fig:#1}}
\newcommand{\Fig}[1]{\figurename~\figref{#1}}
\newcommand{\fig}[1]{\Fig{#1}}
\newcommand{\figlabel}[1]{\label{fig:#1}}
\newcommand{\Tab}[1]{Table~\ref{tab:#1}}
\newcommand{\tab}[1]{\Tab{#1}}
\newcommand{\tablabel}[1]{\label{tab:#1}}
\newcommand{\Eq}[1]{Equation~(\ref{eq:#1})}
\newcommand{\eq}[1]{\Eq{#1}}
\newcommand{\eqalt}[1]{Equation~\ref{eq:#1}}
\newcommand{\eqlabel}[1]{\label{eq:#1}}
\newcommand{\sectionname}{Section}
\newcommand{\Sect}[1]{\sectionname~\ref{sect:#1}}
\newcommand{\sect}[1]{\Sect{#1}}
\newcommand{\sectalt}[1]{\ref{sect:#1}}
\newcommand{\App}[1]{Appendix~\ref{sect:#1}}
\newcommand{\app}[1]{\App{#1}}
\newcommand{\sectlabel}[1]{\label{sect:#1}}

\newcommand{\BIC}{{\ensuremath{\mathrm{BIC}}}}
\newcommand{\TIC}{{\ensuremath{\mathrm{TIC}}}}
\newcommand{\T}{\ensuremath{\mathrm{T}}}
\newcommand{\dd}{\ensuremath{\,\mathrm{d}}}
\newcommand{\bvec}[1]{{\ensuremath{\boldsymbol{#1}}}}
\newcommand{\appropto}{\mathrel{\vcenter{
  \offinterlineskip\halign{\hfil$##$\cr
    \propto\cr\noalign{\kern2pt}\sim\cr\noalign{\kern-2pt}}}}}
\newcommand{\densityunit}{{\ensuremath{\mathrm{nat}^{-2}}}}

% TO DOS
\newcommand{\todo}[3]{{\color{#2}\emph{#1}: #3}}
\newcommand{\dfmtodo}[1]{\todo{DFM}{red}{#1}}
\newcommand{\hoggtodo}[1]{\todo{HOGG}{blue}{#1}}

% Notation for this paper.
\newcommand{\flux}{{\ensuremath{f}}}
\newcommand{\ferr}{{\ensuremath{\sigma_\flux}}}
\newcommand{\attime}{{\ensuremath{t}}}
\newcommand{\basis}{{\bvec{A}}}
\newcommand{\weights}{{\bvec{w}}}

\newcommand{\period}{{\ensuremath{P}}}
\newcommand{\phase}{{\ensuremath{T^0}}}
\newcommand{\duration}{{\ensuremath{D}}}
\newcommand{\depth}{{\ensuremath{Z}}}
\newcommand{\transittime}{{\ensuremath{T}}}
\newcommand{\impact}{{\ensuremath{b}}}
\newcommand{\ecc}{{\ensuremath{e}}}
\newcommand{\pomega}{{\ensuremath{\omega}}}

\newcommand{\datareleaseurl}{{\url{http://bbq.dfm.io/ketu}}}

\begin{document}

\title{%
    Searching for long-period transiting planets in the \kepler\ light curves
    using supervised classification
}

\newcommand{\nyu}{2}
\newcommand{\cds}{3}
\newcommand{\mpia}{4}
\newcommand{\mpis}{5}
\author{%
    Daniel~Foreman-Mackey\altaffilmark{1,\nyu,\cds},
    David~W.~Hogg\altaffilmark{\nyu,\mpia,\cds},
    Bernhard~Sch\"olkopf\altaffilmark{\mpis},
    \etal
}
\altaffiltext{1}         {To whom correspondence should be addressed:
                          \url{danfm@nyu.edu}}
\altaffiltext{\nyu}      {Center for Cosmology and Particle Physics,
                          Department of Physics, New York University,
                          4 Washington Place, New York, NY, 10003, USA}
\altaffiltext{\cds}      {Center for Data Science, New York University,
                          726 Broadway, 7th Floor, New York, NY, 10003, USA}
\altaffiltext{\mpia}     {Max-Planck-Institut f\"ur Astronomie,
                          K\"onigstuhl 17, D-69117 Heidelberg, Germany}
\altaffiltext{\mpis}     {Max Planck Institute for Intelligent Systems
                          Spemannstrasse 38, 72076 T\"ubingen, Germany}

\begin{abstract}

Many of the most dynamically interesting planets will be on orbits longer
than the baseline of existing transit surveys (\kepler, \KT).
Future surveys (\tess, PLATO?) will have such a short continuous coverage that
even habitable zone planets around M-dwarfs will only present a single transit
signal.
Searches for these transiting planets are plagued by false
signals---especially when pushed to low signal-to-noise---and statistical
studies of their population are complicated by weak constraints on the
physical parameters of the system and high rates of false positives.
We develop and present a computationally expensive but tractable method of
searching for single transits using supervised classification methods from
the machine learning literature.
For each month of photometry, we train a random forest classifier on a large
number of simulated signals injected into the photometry of the same star at
different times.
Then, the model's prediction as a function of transit time is evaluated and
times above a threshold probability---chosen for each star to yield a sample
with $\sim99$~percent purity---are given as candidate transits.
Each candidate must then pass a final round of vetting including BLAH BLAH
BLAH.
Applied to every light curve in the \kepler\ archive this method yields a
list of XXXX long-period transiting planet candidates.
Using an informative prior on their eccentricities, we derive weak constraints
on the orbital periods of the candidates and demonstrate that this puts an
upper limit of ZZZZ on the occurrence rate of these planets.

\end{abstract}

\keywords{%
methods: data analysis
---
methods: statistical
---
catalogs
---
planetary systems
---
stars: statistics
}

\section{Introduction}

\begin{itemize}

\item Description of the physical significance of these planets

\item Discussion of previous attempts (papers by Gaudi, Yee, Payne, Bakos,
\etc).

\item Comparison of method to filtering methods and an argument for why
supervised classification should be more robust to false alarms given a large
enough training set

\end{itemize}

\citep{kois}

\section{Data preparation}

The \kepler\ Mission measured photometry for about 150,000 stars at half-hour
cadence for a baseline of over four years.
We aim to search these time series for single transits of long-period planets
and single eclipses of binary stars.
These data are made available on
MAST\footnote{\url{https://archive.stsci.edu/kepler/}} and, for each target,
we downloaded the full set of light curve files provided by Data Release 24
\citep{Thompson:2015}.
From these files, we extracted the PDC time series and split them into
``sections'' with no more than two contiguous missing or flagged data points.
The PDC light curves have been corrected for the instrumental effects caused
by the spacecraft using a data-driven model of the focal plane
\citep{Stumpe:2012, Smith:2012}.
Crucially, an attempt is also made by the PDC procedure to remove sharp
instrumental artifacts like ``sudden pixel sensitivity dropouts (SPSDs)''.
Any remaining missing data points are filled in using linear interpolation
resulting in sections of contiguous time series measurements with nearly
uniform time sampling.
This procedure yields about 150 \emph{light curve sections} for the typical
\kepler\ target.

The goal of this project is to discover the transits of long-period planets
that have not yet been discovered.
Therefore, when studying the light curve of an eclipsing binary star or a star
with known transiting planet candidates---on shorter periods---we also remove
all the in-transit data for the candidate using the parameters provided by the
\project{NASA Exoplanet
Archive}\footnote{\url{http://exoplanetarchive.ipac.caltech.edu/}; We
downloaded the \texttt{cumulative} table of \kepler\ Objects of Interest on
2015-03-25.}.

As we discuss further below, a key assumption of our analysis is that the
process generating any non-transit variability or noise in the light curve
is stationary.

Every three months, the \kepler\ spacecraft rolled 90~degrees to maintain its
orientation with respect to the Sun.
This means that each star landed on a different set of pixels each quarter
and, therefore, the systematic signals caused by any specific spacecraft
orientation won't be shared between neighboring quarters.
A year later, however, the pointing would return to nearly the same setup and
each star ends up back on many of the same pixels.
This manifests itself in the light curves by causing the observations
separated by a year to be qualitatively similar.

For reasons that will become clear in \sect{search}, we split the light curve
sections into 3 disjoint sets.
We use these sets to run a train/validate/test procedure so the goal is to
have representative samples in each set.
Therefore, we attempt to split the sections intelligently; evenly
distributing sections from the same quarter and season across the splits.
When this is not possible, we randomly assign the section to a split.


\section{Random forest classification}

A common task in the machine learning literature is called \emph{supervised
classification} where the goal is to separate objects into classes
represented by sets of labeled examples.
One example from the astronomy literature is the problem of star--galaxy
separation (\dfmtodo{cite}).
In this problem, there are a set of images or photometric measurements that
have been classified---by some other method---as either a star or galaxy
observation and the goal is to transfer these labels to a set of observations
that have not yet been classified.
For a more in-depth discussion of the application of these techniques in
astronomy, the interested reader is directed to \citet{Ivezic:2013}.

The Random Forest (RF) classification model \citep{Breiman:2001} has become
very popular in the machine learning community.
It has also been applied with great success in astronomy \citep[for
example]{Richards:2011, Richards:2012} \dfmtodo{others}.
The RF model works by building




\section{Search procedure}\sectlabel{search}

\paragraph{Generate the training \& validation sets}

\paragraph{Train}

\paragraph{Validate}

\paragraph{Test}


\section{Tuning parameters}

\paragraph{The parameters}

\paragraph{Objective function}


\section{Demonstration}

\paragraph{Successes}

\paragraph{Failures}


\section{Parameter estimation}


\section{Population inference}


\section{Results}


\section{Discussion}


\acknowledgments
It is a pleasure to thank
\ldots
for helpful contributions to the ideas and code presented here.
DFM and DWH were partially supported by the National Science Foundation
(grant IIS-1124794),
the National Aeronautics and Space Administration
(grant NNX12AI50G), and the Moore--Sloan Data Science Environment at NYU.

This research made use of the NASA \project{Astrophysics Data System} and the
NASA Exoplanet Archive.
The Archive is operated by the California Institute of Technology, under
contract with NASA under the Exoplanet Exploration Program.
This \paper\ includes data collected by the \kepler\ mission. Funding for the
\kepler\ mission is provided by the NASA Science Mission directorate.
We are grateful to the entire \kepler\ team, past and present.
Their tireless efforts were all essential to the tremendous success of the mission
and the successes of \KT, present and future.
These data were obtained from the Mikulski Archive for Space Telescopes
(MAST).
STScI is operated by the Association of Universities for Research in
Astronomy, Inc., under NASA contract NAS5-26555.
Support for MAST is provided by the NASA Office of Space Science via grant
NNX13AC07G and by other grants and contracts.

{\it Facilities:} \facility{Kepler}

\appendix

\section{Some appendix}

\clearpage
\bibliography{peerless}
\clearpage


% \begin{figure}[p]
% \begin{center}
% \includegraphics{figures/pca.pdf}
% \end{center}
% \caption{%
% The top 10 eigen light curves (ELCs) generated by running principal component
% analysis on all the aperture photometry from Campaign~1.
% \figlabel{pca}}
% \end{figure}

\end{document}
