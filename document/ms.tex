% This paper is part of the single transits project.
% Copyright 2015 Dan Foreman-Mackey (NYU) and the co-authors listed below.
%
%  RULES OF THE GAME
%
%  * 80 characters
%  * line breaks at the ends of sentences
%  * eqnarrys ONLY
%  * ``light curve'' not ``light-curve'' or ``lightcurve''
%  * Do not put in any comments that might get tweeted by @OverheardOnAph
%    (or maybe do put in a few....)
%  * ``percent'' (not \%) is a unit, as is ppm, so 5~percent.
%  * that is all.
%

\documentclass[12pt,preprint]{aastex}

\pdfoutput=1

\usepackage{color,hyperref}
\definecolor{linkcolor}{rgb}{0,0,0.5}
\hypersetup{colorlinks=true,linkcolor=linkcolor,citecolor=linkcolor,
            filecolor=linkcolor,urlcolor=linkcolor}
\usepackage{url}
\usepackage{amssymb,amsmath}
\usepackage{subfigure}
\usepackage{booktabs}

\usepackage{natbib}
\bibliographystyle{apj}

\newcommand{\project}[1]{\textsl{#1}} % hogg say
\newcommand{\kepler}{\project{Kepler}}
\newcommand{\KT}{\project{K2}}
\newcommand{\tess}{\project{TESS}}
\newcommand{\jwst}{\project{JWST}}
\newcommand{\terra}{\project{TERRA}}
\newcommand{\pdc}{\project{PDC}}
\newcommand{\license}{MIT License}
\newcommand{\projectname}{\project{ketu}}

\newcommand{\paper}{\textsl{Article}}

\newcommand{\foreign}[1]{\emph{#1}}
\newcommand{\etal}{\foreign{et\,al.}}
\newcommand{\etc}{\foreign{etc.}}
\newcommand{\True}{\foreign{True}}
\newcommand{\Truth}{\foreign{Truth}}

\newcommand{\figref}[1]{\ref{fig:#1}}
\newcommand{\Fig}[1]{\figurename~\figref{#1}}
\newcommand{\fig}[1]{\Fig{#1}}
\newcommand{\figlabel}[1]{\label{fig:#1}}
\newcommand{\Tab}[1]{Table~\ref{tab:#1}}
\newcommand{\tab}[1]{\Tab{#1}}
\newcommand{\tablabel}[1]{\label{tab:#1}}
\newcommand{\Eq}[1]{Equation~(\ref{eq:#1})}
\newcommand{\eq}[1]{\Eq{#1}}
\newcommand{\eqalt}[1]{Equation~\ref{eq:#1}}
\newcommand{\eqlabel}[1]{\label{eq:#1}}
\newcommand{\sectionname}{Section}
\newcommand{\Sect}[1]{\sectionname~\ref{sect:#1}}
\newcommand{\sect}[1]{\Sect{#1}}
\newcommand{\sectalt}[1]{\ref{sect:#1}}
\newcommand{\App}[1]{Appendix~\ref{sect:#1}}
\newcommand{\app}[1]{\App{#1}}
\newcommand{\sectlabel}[1]{\label{sect:#1}}

\newcommand{\BIC}{{\ensuremath{\mathrm{BIC}}}}
\newcommand{\TIC}{{\ensuremath{\mathrm{TIC}}}}
\newcommand{\T}{\ensuremath{\mathrm{T}}}
\newcommand{\dd}{\ensuremath{\,\mathrm{d}}}
\newcommand{\bvec}[1]{{\ensuremath{\boldsymbol{#1}}}}
\newcommand{\appropto}{\mathrel{\vcenter{
  \offinterlineskip\halign{\hfil$##$\cr
    \propto\cr\noalign{\kern2pt}\sim\cr\noalign{\kern-2pt}}}}}
\newcommand{\densityunit}{{\ensuremath{\mathrm{nat}^{-2}}}}

% TO DOS
\newcommand{\todo}[3]{{\color{#2}\emph{#1}: #3}}
\newcommand{\dfmtodo}[1]{\todo{DFM}{red}{#1}}
\newcommand{\hoggtodo}[1]{\todo{HOGG}{blue}{#1}}

% Notation for this paper.
\newcommand{\flux}{{\ensuremath{f}}}
\newcommand{\ferr}{{\ensuremath{\sigma_\flux}}}
\newcommand{\attime}{{\ensuremath{t}}}
\newcommand{\basis}{{\bvec{A}}}
\newcommand{\weights}{{\bvec{w}}}

\newcommand{\period}{{\ensuremath{P}}}
\newcommand{\phase}{{\ensuremath{T^0}}}
\newcommand{\duration}{{\ensuremath{D}}}
\newcommand{\depth}{{\ensuremath{Z}}}
\newcommand{\transittime}{{\ensuremath{T}}}
\newcommand{\impact}{{\ensuremath{b}}}
\newcommand{\ecc}{{\ensuremath{e}}}
\newcommand{\pomega}{{\ensuremath{\omega}}}

\newcommand{\datareleaseurl}{{\url{http://bbq.dfm.io/ketu}}}

\begin{document}

\title{%
    The search for long-period transiting planets in the \kepler\ light curves
}

\newcommand{\nyu}{2}
\newcommand{\mpia}{3}
\newcommand{\cds}{4}
\newcommand{\mpis}{5}
\author{%
    Daniel~Foreman-Mackey\altaffilmark{1,\nyu},
    David~W.~Hogg\altaffilmark{\nyu,\mpia,\cds},
    Bernhard~Sch\"olkopf\altaffilmark{\mpis},
    \etal
}
\altaffiltext{1}         {To whom correspondence should be addressed:
                          \url{danfm@nyu.edu}}
\altaffiltext{\nyu}      {Center for Cosmology and Particle Physics,
                          Department of Physics, New York University,
                          4 Washington Place, New York, NY, 10003, USA}
\altaffiltext{\mpia}     {Max-Planck-Institut f\"ur Astronomie,
                          K\"onigstuhl 17, D-69117 Heidelberg, Germany}
\altaffiltext{\cds}      {Center for Data Science, New York University,
                          726 Broadway, 7th Floor, New York, NY, 10003, USA}
\altaffiltext{\mpis}     {Max Planck Institute for Intelligent Systems
                          Spemannstrasse 38, 72076 T\"ubingen, Germany}

\begin{abstract}

Many of the most dynamically interesting planets will be on orbits longer
than the baseline of existing transit surveys (\kepler, \KT).
Future surveys (\TESS, PLATO?) will have such a short continuous coverage that
even habitable zone planets around M-dwarfs will only present a single transit
signal.
Searches for these transiting planets are plagued by false
signals---especially when pushed to low signal-to-noise---and statistical
studies of their population are complicated by weak constraints on the
physical parameters of the system and high rates of false positives.
We develop and present a computationally expensive but tractable method of
searching for single transits using supervised classification methods from
the machine learning literature.
For each month of photometry, we train a random forest classifier on a large
number of simulated signals injected into the photometry of the same star at
different times.
Then, the model's prediction as a function of transit time is evaluated and
times above a threshold probability---chosen for each star to yield a sample
with $\sim99$~percent purity---are given as candidate transits.
Each candidate must then pass a final round of vetting including BLAH BLAH
BLAH.
Applied to every light curve in the \kepler\ archive this method yields a
list of XXXX long-period transiting planet candidates.
Using an informative prior on their eccentricities, we derive weak constraints
on the orbital periods of the candidates and demonstrate that this puts an
upper limit of ZZZZ on the occurrence rate of these planets.

\end{abstract}

\keywords{%
methods: data analysis
---
methods: statistical
---
catalogs
---
planetary systems
---
stars: statistics
}

\section{Introduction}

Hi \citep{kois}.

\acknowledgments
It is a pleasure to thank
\ldots
for helpful contributions to the ideas and code presented here.
DFM and DWH were partially supported by the National Science Foundation
(grant IIS-1124794),
the National Aeronautics and Space Administration
(grant NNX12AI50G), and the Moore--Sloan Data Science Environment at NYU.

This research made use of the NASA \project{Astrophysics Data System} and the
NASA Exoplanet Archive.
The Archive is operated by the California Institute of Technology, under
contract with NASA under the Exoplanet Exploration Program.
This \paper\ includes data collected by the \kepler\ mission. Funding for the
\kepler\ mission is provided by the NASA Science Mission directorate.
We are grateful to the entire \kepler\ team, past and present.
Their tireless efforts were all essential to the tremendous success of the mission
and the successes of \KT, present and future.
These data were obtained from the Mikulski Archive for Space Telescopes
(MAST).
STScI is operated by the Association of Universities for Research in
Astronomy, Inc., under NASA contract NAS5-26555.
Support for MAST is provided by the NASA Office of Space Science via grant
NNX13AC07G and by other grants and contracts.

{\it Facilities:} \facility{Kepler}

\appendix

\section{Some appendix}

\clearpage
\bibliography{single}
\clearpage


% \begin{figure}[p]
% \begin{center}
% \includegraphics{figures/pca.pdf}
% \end{center}
% \caption{%
% The top 10 eigen light curves (ELCs) generated by running principal component
% analysis on all the aperture photometry from Campaign~1.
% \figlabel{pca}}
% \end{figure}

\end{document}
